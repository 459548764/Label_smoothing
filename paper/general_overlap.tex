\section{General theory for Overlapping Otherwise Separable classes (OOSe)}
Let $\mathcal X$ be a measurable space
% and let $\mathbb P$ be a probability distribution on the product space
% $\mathcal X \times \{\pm 1\}$,
and let $h_{\text{true}}:\mathcal X \rightarrow \{\pm 1\}$ be a measurable
function. Let $X$ be a random variable on $\mathcal X$ and $Y$ be a random
variable on $\{\pm 1\}$ such that the joint distribution of $(X,Y)$ is $\mathbb
P$. Let $C$ be a measurable subset of $\mathcal X$ and define
$\alpha:=\mathbb{P}(X \in C)$. Finally, define $\beta := \mathbb P(Y=1|X\in C)$
and $B:=\mathcal X\setminus C$. Suppose the following conditions hold

\begin{assumption}[Overlapping Otherwise Separable (OOSe)]
\begin{eqnarray}
  \begin{split}
    &\mathbb P(Y=y|X=x)=\delta_{h_{\text{true}}(x)}(y), \forall x \in B,\\    
    &\mathbb P(Y=1|X=x) < 1/2,\;\forall x \in C.
  \end{split}
\end{eqnarray}
\label{thm:overlap_assumption}
\end{assumption}
\begin{mdframed}
  \textbf{The elephant in the room.}
  Intuitively, for a given $x \in \mathcal X$, one should think of the
distribution of $Y|x$ as a binary measurement on the random variable $X$, which
equals $h(x)$ under normal circumstances, but degrades to $\bern(\{\pm
1\};\beta)$ noise when $X$ hits the set $C$. Thus $C$ should be thought of as
``large'' (according to how $\alpha$ is big) set of examples with noisy /
overlapping labels.
\end{mdframed}

\subsection{Bayes optimal error rate}
We now show that the Bayes optimal
error classification problem defined by the joint distribution $\mathbb P$ is
$\alpha\beta$. Indeed, let $h^*(x) = \text{sign}(\mathbb{P}(Y=1|X=x)- 1/2)$ be
the Bayes optimal classifier. A direct computation yields
\begin{eqnarray}
  \label{eq:bayesh}
h^*(x) = \begin{cases}1,&\mbox{ if }x \in B\text{ and
  }h_{\text{true}}(x)=1,\\-1,&\mbox{ else.}\end{cases}
\end{eqnarray}
Note that $h^*$ agrees perfectly with $h$ on the set $B$, i.e
$h^*(x)=h_{\text{true}}(x),\;\forall x \in B$. Now, one can compute the
accuracy of the Bayes optimal classifier $h^*$ as
\begin{eqnarray*}
  \begin{split}
    \acc(h)&:=\mathbb{P}(h^*(X)=Y)=(1-\alpha)
    \mathbb{P}(h^*(X)=h_{\text{true}}(X)|X \in B) + \alpha \mathbb{P}(h^*(X)=Y|X
    \in C)\\
    &=(1-\alpha)(1) + \alpha \mathbb{P}(Y=-1|X \in C)=1-\alpha + \alpha
    \mathbb{P}(Y=-1|X\in C)\\
    &=1-\alpha + \alpha(1-\beta)= 1 - \alpha\beta=1-\mathbb P(X \in C,Y=1).
  \end{split}
\end{eqnarray*}
Thus no classifier can attain an accuracy larger than $1-\alpha\beta$ on this
problem! BTW, a simple application of Bayes rule
\begin{eqnarray}
  \alpha\beta=\mathbb P(X \in C)\mathbb P(Y=1|X\in C)
  =\mathbb P(X \in C,Y=1)
  =\mathbb P(Y=1) \mathbb P(X \in C|Y=1).
\end{eqnarray}

\subsection{Bounds on adversarial robustness of Bayes optimal classifier}
Suppose $\mathcal X=(\mathcal X,d)$ is a  metric space and recall the
definition of the sets $C$ and $B$ above. Let $d(x,B):=\inf_{x' \in B}d(x,x')$
denote the distance of a point $x \in \mathcal X$ from a subset $A$ of
$\mathcal X$.
Note that $d(x,A_1 \cup A_2) \le \min(d(x,A_2),d(x,A_2))$ and $d(x,A_1 \cap A_2)
\ge \max(d(x,A_2),d(x,A_2))$.

For a measurable function $h:\mathcal X \rightarrow \{\pm 1\}$ (aka a classifier)
and for any $y \in \{\pm 1\}$, define $B_y(h) = \{x \in \mathcal X | h(x) \ne
y\}$ and observe that the adversarial robustness accuracy of $h$ on the class
$y$ can be written as $\acc_\epsilon(h|y)=\mathbb P_{X|y}(d(X,B_y(h)) \ge
\epsilon)$. Also note that there is a one-to-one correspondence between a
classifier $h$ and the set $B_1(h)$ (or $B_{-1}(h)$). At a fixed level of
standard accuracy, finding the most robustness classifier then amounts to
finding the surface-minimizing shape $B_1(h)$ in the measure space $(\mathcal X,
\mathbb P_{X|1})$. This is a purely geometric problem!
\begin{mdframed}
  \textbf{The grand schema of things.}
  Lemma \ref{thm:bayesrob} below is the first in a series which will give us
  powerful backdoors for understanding the link between class overlap,
  label-smoothing, and adversarial robustness. Also, we can already smell
  ``measure concentration'' in the air, and so one could hope for adversarial
  robustness ``No Free Lunch'' theorems (in the spirit of
  ~\citep{dohmatob2018limitations}) for Bayes optimal classifiers and nearest
  neighbor classifiers...
\end{mdframed}


\begin{lemma}
For every $\epsilon > 0$, the adversarial robustness accuracy of the Bayes
optimal classifier is bounded as follows
\begin{itemize}
\item $\acc_\epsilon(h^*|-1) \ge \mathbb P_{X|-1}(d(X,B)\ge \epsilon)$, and
\item $\acc_\epsilon(h^*|1) \le \mathbb P_{X|1}(d(X,C)\ge \epsilon)$.
\end{itemize}
Moreover, these inequalities are equalities if $\mathbb P(h_{\text{true}}(X)=-1|X
\in C)=1$.
\label{thm:bayesrob}
\end{lemma}
\begin{proof}
  Now, by
  formula \eqref{eq:bayesh} defining $h^*$, one has $B_{-1}(h^*) \subseteq B$
  and $C \subseteq B_1(h^*)$, and so
  \begin{eqnarray}
    d(x,B_{-1}(h^*)) \ge d(x,B)\text{ and }d(x,B_1(h^*)) \le d(x,C),\;\forall x
    \in \mathcal X
  \label{eq:dist}
  \end{eqnarray}
  Thus
  \begin{eqnarray*}
    \begin{split}
    \acc_\epsilon(h^*|-1)&=\mathbb P_{X|-1}(d(X,B_{-1}(h^*)) \ge \epsilon) \ge
    \mathbb P_{X|-1}(d(X,B)\ge \epsilon),\text{ and}\\
    \acc_\epsilon(h^*|1)&=\mathbb P_{X|1}(d(X,B_{1}(h^*)) \ge \epsilon) \le
    \mathbb P_{X|1}(d(X,C)\ge \epsilon).
    \end{split}
  \end{eqnarray*}

  Finally, if $\mathbb P(h_{\text{true}}(X)=-1|X \in C)=1$, then inequalities
  are actually equalities almost surely, and so the above inequalities for the
  adversarial robustness accuracies are also equalities.
\end{proof}

\subsubsection{Applications (special cases)}
\paragraph*{The example from section \ref{sec:overlap_example} revisited.} Lets
demonstrate a another approach to obtain the results from the example in section
\ref{sec:overlap_example}. Using the above reasoning, one can immediately
recover the result $\acc(h)=1-\gamma/2$. Indeed, in that example, one had
\begin{itemize}
\item $\mathcal X = [-1,1]$, $C=[-1,0)$, $B=[0,1]$, $h=\text{sign}(\cdot)$.
\item For all $x \in C$, using Bayes rule, we have
  \begin{eqnarray*}
    \begin{split}
      \mathbb P(Y=1|X=x) &= \frac{\gamma f_{X|1}(x)}{\gamma
    f_{X|1}(x)+(1-\gamma)f_{X|-1}(x)}=\frac{\gamma (1+x)}{\gamma
    (1+x)+2(1-\gamma)(1+x)}\\
  &=\gamma/(2-\gamma) \in [1/3,1/2],\text{ since }\gamma \in [1/2,2/3].
\end{split}
  \end{eqnarray*}
\end{itemize}
Thus Assumption \eqref{thm:overlap_assumption} is met. Still using Bayes rule,
one computes
$$
\alpha:=\mathbb P(X \in C)=\mathbb P(X < 0) = \gamma
F_{X|1}(0)+(1-\gamma)F_{X|-1}(0)=\gamma/2 + (1-\gamma)=1-\gamma/2.
$$
Likewise,
$$
\beta := \mathbb P(Y=1|X<0)=\frac{\gamma\mathbb
  P(X<0|Y=1)}{\mathbb P(X<0)}=(\gamma/2)/\alpha=\gamma/(2-\gamma),
$$
and $\alpha\beta=\alpha(\gamma/2)/\alpha=\gamma/2$.
$\therefore \acc(h^*)=1-\alpha\beta=1-\gamma/2$.

As regards adversarial robustness accuracy, consider  the absolute value metric
$d(x,x'):=|x-x'|$ on $\mathcal X:=[-1,1]$. For all $x \in \mathcal X$, one
easily computes $d(x,C)=x\mathbb
1_{[0,1]}(x)=(x)_+$ and $d(x,B)=-x\mathbb 1_{[-1, 0]}(x)=(-x)_+$.
For every $\epsilon \ge 0$, invoking Lemma \ref{thm:bayesrob} then gives 
\begin{eqnarray*}
  \begin{split}
    &\acc_\epsilon(h^*|1)=\mathbb P_{X|1}(d(x,C) \ge \epsilon)=\int_{x \in \mathcal X,
      d(x,C)\ge\epsilon}f_{X|1}(x)dx\\
    &=\int_{x \in [-1, 1],\;(x)_+\ge\epsilon}f_{X|1}(x)dx
    =\int_{\epsilon}^1f_{X|1}(x)dx=\int_{\epsilon}^1(1-x)dx\\
    &=1-1^2/2 - (\epsilon - \epsilon^2/2)= (1-\epsilon)^2/2,
\end{split}
  \end{eqnarray*}
and likewise
\begin{eqnarray*}
  \begin{split}  
    &\acc_\epsilon(h^*|-1)=\mathbb P_{X|-1}(d(x,B) \ge \epsilon)=\int_{x \in
      \mathcal X, d(x,B)\ge\epsilon}f_{X|-1}(x)dx\\
    &=\int_{x \in [-1, 1],\;(-x)_+\ge\epsilon}f_{X|-1}(x)dx
    =\int_{-1}^{-\epsilon}f_{X|-1}(x)dx=\int_{-1}^{-\epsilon}2(1+x)dx\\
    &= -2\epsilon+\epsilon^2+2-1 = (1-\epsilon)^2.
\end{split}
\end{eqnarray*}
Furthermore, one can then compute the total adversarial robustness accuracy as
$$\acc_\epsilon(h)=\gamma(1-\epsilon)^2/2 + (1  - \gamma)(1-\epsilon)^2
=(1-\gamma/2)(1-\epsilon)^2,
$$
which is precisely the expression obtained ``manually'' in
\ref{sec:overlap_example}.

%=1-\alpha\beta-(1-\beta)\mathbb P_X(d(X,C)>\epsilon)$