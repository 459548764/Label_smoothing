\section{Contributions}
\subsection{Adversarial label smoothing}
Let $\tv(\cdot\|\cdot)$ be the Total-Variation distance  defined by $\tv(q\|p)
:= (1/2)\|q-p\|_1$. Let $\hat{P}_n := \frac{1}{n}\sum_{i=1}\delta_{x_i} \otimes
\delta_{y_i}$ be the empirical distribution of the input-label pairs $(x,y)$ in
the dataset. For an uncertainty tolerance level $\epsilon>0$, define the
uncertainty set
\begin{eqnarray}
  \begin{split}
    \mathcal U_\epsilon(\hat{P}_n) &:= \{P_XQ_{Y|X} |
  \tv(Q_{Y|x}\|\hat{P}_n(\cdot|x)) \le \epsilon,\;\forall x \in \mathcal X\}\\
  &=\left\{\sum_{i=1}^n(1/n)\delta_{x_i} \otimes q_i | q_i \in \Delta_k,\;
    \tv(q_i\|\delta_{y_i})\le \epsilon\;\forall i \in [\![n]\!]\right\},
  \end{split}
\end{eqnarray}
made up of joint distributions $Q \in \mathcal P(\mathcal X \times \mathcal Y)$
on input-label pairs $(x_i,y_i)$, for which the marginal label distribution
$q_i:=Q_{Y|X=x_i} \in \mathcal P(\mathcal Y)=\Delta_k$ is within  TV distance
$\le \epsilon$ of the one-hot encoding of the observed label $\delta_{y_i}$. By
direct computation, one has
$$
\tv(q_i\|\delta_{y_i}) = (1/2)\sum_{j=1}^k|q_i(j)-\delta_{y_i}(j)| = (1/2)
\left(\sum_{j \ne y_i}q_i(j) + 1 - q_i(y_i)\right)=1-q_i(y_i),
$$
and so the uncertainty set can be rewritten as
\begin{eqnarray}
  \mathcal U_\epsilon(\hat{P}_n)=\left\{\sum_{i=1}^n(1/n)\delta_{x_i} \otimes q_i | q_i \in
  \Delta_k,\;q_i(y_i)\ge 1 - \epsilon\;\forall i \in [\![n]\!]\right\}.
\end{eqnarray}



Now, consider the two-player game
\begin{eqnarray}
  \begin{split}
    \min_{\theta}\max_{Q \in \mathcal U_\epsilon(\hat{P}_n)} \mathbb E_{(x,y)
      \sim Q}[-\delta_y^T\log(p(\cdot|x;\theta))].
  \end{split}
  \label{eq:grand}
\end{eqnarray}
Mindful of the previous computations, one has
\begin{eqnarray}
  \begin{split}
    \min_{\theta}\max_{Q \in \mathcal U_\epsilon(\hat{P}_n)} \mathbb E_{(x,y) \sim
      Q}[-\delta_y^T\log(p(x;\theta))]
    =\min_{\theta}\frac{1}{n}\sum_{i=1}^n\max_{q_i \in
      \Delta_k|q_i(y_i) \ge 1-\alpha} -q_i^T\log(p(x_i;\theta)),
    \end{split}
\end{eqnarray}
where  $\alpha := \min(\epsilon,1) \in [0, 1]$. Thanks to Lemma
\ref{thm:bumbednash} (see Appendix \ref{sec:proofs}), the inner problem in the above
display has analytic solution given by
\begin{mdframed}
  \textbf{Adversarial Label-Smoothing (ALS).}
\begin{eqnarray}
  q_i^*(\cdot|\theta) = (1-\alpha) \delta_{y_i} +
  \alpha\text{ConvHull}(\argmin_{j=1}^kz(j|x_i;\theta))).
  \label{eq:ls}
\end{eqnarray}
\end{mdframed}
The interpretation is intuitive. $\alpha$ acts as a smoothing parameter.
If $\alpha = 1$, then the adversarial weights $q^*_i(\cdot|\theta)$ wanders freely
in the sub-simplex spanned by the smallest components of the logit vector
$z(\cdot|x_i;\theta)$. If $\alpha=0$, then $q^*_i(\cdot|\theta)=\delta_{y_i}$,
and we back to the unrobustified problem. Thus \eqref{eq:ls} is a generalization
of the classical label-smoothing idea ~\citep{labelsmoothing,labelsmoothingbis}\todo{XXX cite
  goodfellow and clique}. For $0 < \alpha < 1$, $q^*_i(\cdot|\theta)$ is a
proper convex combination of the two previous cases.

The grand problem \eqref{eq:grand} then reduces to
\begin{eqnarray}
  \begin{split}
    \min_{\theta}\frac{1}{n}\sum_{i=1}^n
    -q^*_i(\cdot|\theta)^T\log(p(\cdot|x_i;\theta))&=
    \min_{\theta}\frac{1}{n}\sum_{i=1}^n
    \text{CrossEntropy}(p(\cdot|x_i;\theta),q^*_i(\cdot|\theta))\\
    &=\min_{\theta}(1-\alpha)L_n(\theta)+\alpha R_n(\theta),
    \end{split}
\end{eqnarray}
where
\begin{itemize}
  \item $L_n(\theta) :=
    \frac{1}{n}\sum_{i=1}^n\text{CrossEntropy}(p(\cdot|x_i;\theta),\delta_{y_i}) \ge 0$ is
    the usual loss without label-smoothing, and
  \item  $R_n(\theta) := \frac{1}{n}\sum_{i=1}^n\max_{j=1}^k
    -\log(p(j|x_i;\theta)) \ge 0$.
    \end{itemize}
The term $R_n(\theta)$ acts as a penalty which implements some kind of
\textit{logit squeezing}. Indeed, for each data point $x_i$ with true label
$y_i$, it forces the model to put non-negligible mass on each other label $j \ne
y_i$. To see this, note that if $p(j|x_i;\theta) < e^{-\gamma}$ for some
$\gamma > 0$, then $R_n \ge \alpha \gamma$, and $R_n(\theta) \overset{\gamma
  \rightarrow \infty}{\longrightarrow} \infty$ (provided $\alpha > 0$).


\subsection{Boltzmann Label-Smoothing (BLS)}
According to \eqref{eq:ls}, excluding the special case where several class
labels have the smallest logit values, adversarial label smoothing thus gives
weights to only two class labels: the true class label (due to the constraint of
the model) and the class label which minimizes the logit vector. Adversarial
labels smoothing thus gives "two-hot" labels rather than really "smoothed"
labels. Replace \emph{hard-min} in \eqref{eq:ls}  with a \emph{softmin}. This
leads to what we call \emph{Boltzmann label distribution} (BLS), defined
by
$$
q^{\text{B}}_i = (1- \alpha) \delta_{y_i} + \alpha \text{Boltz}_T(\cdot|x_i;\theta)
$$
where
\begin{eqnarray}
  \text{Boltz}_T(j|x_i;\theta) = \frac{\exp(-z(j|x_i;\theta)/T)}{\sum_{k=1}^K
  \exp(-z(k|x_i;\theta)/T)}
\end{eqnarray}
is the Boltzmann distribution with energy levels $z(j|x_i;\theta)$ at
temperature $T>0$.
We easily check that $q^{\text{B}}(y) > 1- \alpha$ and $q^{\text{B}}$ is a
probability distribution, thus $q^{\text{B}} \in \mathcal
U_\epsilon(\hat{P}_n)$. More generally, for any probability distribution $q$ on
the space of class labels, $(1- \alpha) \delta_y + \alpha q^{\text{B}}$ is a probability
distribution belonging to $\mathcal U_\epsilon(\hat{P}_n)$.

One notes that, in the LBLS scheme above,
\begin{itemize}
\item The low-temperature limit $T \rightarrow 0^+$ corresponds to ALS
\item The high-temperature limit $T \rightarrow +\infty$ corresponds to standard
  global label-smoothing
\end{itemize}

\input{theoretical_understanding.tex}

\input{intuition.tex}

\subsection{General theory for overlapping classes}
Let $X$ be a random variable with law $\mathbb{P}_X$ on a space $\mathcal X$ and let
$h_{\text{true}}:\mathcal X \rightarrow \{\pm 1\}$ be a measurable function. Let $C \subseteq
\mathcal X$ be an event with probability $\mathbb{P}(x \in C):=\int_{x \in C}d\mathbb{P}_X(x)
=\alpha \in (0, 1]$. Let $\beta \in (0, 1)$ be fixed and define a random variable $Y$
by prescribing the following conditional law $$\mathbb{P}_{Y|x}
= \begin{cases}\bern(\{\pm 1\};\beta),&\mbox{ if }x \in
  C,\\\delta_{h(x)},&\mbox{ else.}\end{cases}
$$
For a given $x \in \mathcal X$, one should think of the distribution of $Y|x$ as
a binary measurement on the random variable $X$, which equals $h(x)$ under
normal circumstances, but degrades to uninformative Bernoulli noise when $X$
hits the set $C$. Let $B:=\mathcal X\setminus C$, $B_y:=h_{\text{true}}^{-1}(y)\cap
B = \{x \in B | h(x)=1\}$, and note that $h$ agrees with
$h_{\text{true}}$ on the event $B$, i.e $h(x)=h_{\text{true}}(x)$ for all $x \in
B$. Let $h^*(x) = \text{sign}(\mathbb{P}(Y=1|X=x)- 1/2)$ be the Bayes optimal
classifier, and w.l.o.g suppose $\beta < 1/2$. Then a direct computation yields
$$
h^*(x) = \begin{cases}1,&\mbox{ if }x \in B_1,\\-1,&\mbox{
    else,}\end{cases}
$$
Thus $\mathbb{P}(h^*(X)=1|X \in B)=\mathbb{P}(h(X)=1|X \in B)$. Now, one can compute the
accuracy of the Bayes optimal classifier $h^*$ as
\begin{eqnarray*}
  \begin{split}
    \acc(h)&:=\mathbb{P}(h^*(X)=Y)=(1-\alpha)
    \mathbb{P}(h^*(X)=h_{\text{true}}(X)|X \in B) + \alpha \mathbb{P}(h^*(X)=Y|X
    \in C)\\
   (1-\alpha)\mathbb{P}((h^*(X)=1|X \in B_1)+(h^*(X)=1|X \in B_{-1})) + \alpha \mathbb{P}(Y=-1|X
    \in C)\\    
    &=(1-\alpha)(1) + \alpha \mathbb{P}(Y=0|X\in C)=1-\alpha + \alpha(1-\beta) = 1 -
    \alpha\beta.
  \end{split}
\end{eqnarray*}
Thus no classifier can attain an accuracy larger that $1-\alpha\beta$ on this problem!

\subsubsection{Things fall apart}
Note that $\acc_\epsilon(h^*) \le 1-\alpha+(1-\beta)\mathbb P_X(d(X,C) \le
\epsilon)$
%=1-\alpha\beta-(1-\beta)\mathbb P_X(d(X,C)>\epsilon)$